\documentclass[12pt,oneside]{memoir}

% ============================================
% Pattern Validation Protocol v3.1
% Complete Implementation Framework
% The Fractality Institute
% ============================================

% --------- Packages ---------
\usepackage[utf8]{inputenc}
\usepackage[T1]{fontenc}
\usepackage{lmodern}
\usepackage{geometry}
\geometry{margin=1in}
\usepackage{setspace}
\onehalfspacing
\usepackage{amsmath, amssymb, amsthm, mathtools}
\usepackage{physics}
\usepackage{bm}
\usepackage{graphicx}
\usepackage{booktabs}
\usepackage{enumitem}
\usepackage{hyperref}
\usepackage{listings}
\usepackage{xcolor}
\lstset{
    language=Python,
    basicstyle=\ttfamily\small,
    commentstyle=\color{gray},
    keywordstyle=\color{blue},
    stringstyle=\color{red},
    numbers=left,
    numberstyle=\tiny\color{gray},
    breaklines=true,
    frame=single,
    captionpos=b
}
\hypersetup{
    colorlinks=true,
    linkcolor=blue!60!black,
    citecolor=blue!60!black,
    urlcolor=blue!60!black,
    pdftitle={Pattern Validation Protocol v3.0},
    pdfauthor={Fractality Institute Methodology Council}
}
\usepackage{algorithm}
\usepackage{algpseudocode}
\usepackage{siunitx}
\usepackage{csquotes}
\usepackage{titlesec}
\usepackage{tocloft}
\usepackage{epigraph}
\usepackage{microtype}
\usepackage{tikz}
\usetikzlibrary{arrows.meta,calc,decorations.pathmorphing,positioning,shapes,patterns}
\usepackage{pgfplots}
\pgfplotsset{compat=1.17}
\usepackage{multirow}
\usepackage{longtable}
\usepackage{array}
\newcolumntype{L}[1]{>{\raggedright\arraybackslash}p{#1}}
\newcolumntype{C}[1]{>{\centering\arraybackslash}p{#1}}
\newcolumntype{R}[1]{>{\raggedleft\arraybackslash}p{#1}}

% --------- Theorem Environments ---------
\theoremstyle{plain}
\newtheorem{theorem}{Theorem}[chapter]
\newtheorem{law}[theorem]{Law}
\newtheorem{principle}[theorem]{Principle}
\newtheorem{proposition}[theorem]{Proposition}
\newtheorem{lemma}[theorem]{Lemma}
\newtheorem{corollary}[theorem]{Corollary}

\theoremstyle{definition}
\newtheorem{definition}[theorem]{Definition}
\newtheorem{axiom}[theorem]{Axiom}
\newtheorem{protocol}[theorem]{Protocol}
\newtheorem{stage}[theorem]{Stage}

\theoremstyle{remark}
\newtheorem{remark}[theorem]{Remark}
\newtheorem{example}[theorem]{Example}
\newtheorem{falsifier}[theorem]{Falsifier}
\newtheorem{warning}[theorem]{Warning}

% --------- Custom Commands ---------
\newcommand{\PVP}{\textsc{PVP}}
\newcommand{\FI}{\textsc{FI}}
\newcommand{\UCT}{\textsc{UCT}}
\newcommand{\BSW}{\textsc{BSW}}
\newcommand{\R}{\mathbb{R}}
\newcommand{\N}{\mathbb{N}}
\newcommand{\Z}{\mathbb{Z}}
\newcommand{\E}{\mathbb{E}}
\DeclareMathOperator{\argmax}{arg\,max}
\DeclareMathOperator{\argmin}{arg\,min}
\newcommand{\BigO}{\mathcal{O}}
\newcommand{\Pattern}{\mathcal{P}}
\newcommand{\Complexity}{\mathcal{K}}
\newcommand{\Crystal}{\mathcal{C}}

% Custom colors
\definecolor{stagecolor}{RGB}{0,100,200}
\definecolor{warncolor}{RGB}{200,100,0}
\definecolor{passcolor}{RGB}{0,150,0}
\definecolor{failcolor}{RGB}{200,0,0}

% --------- Title Setup ---------
\title{\Huge Pattern Validation Protocol v3.1\\[0.5em]
       \LARGE Complete Implementation Framework\\[0.5em]
       \Large With Statistical Rigor and Computational Guidance\\[1em]
       \large Canon I (Empirical) with Canon II/III Cross-Support}
       
\author{\Large Fractality Institute Methodology Council\\[0.5em]
        \normalsize Maintainer: methodology@fractality.institute\\[1em]
        \normalsize Document ID: FI-MP-004-v3.1\\
        \normalsize Release Date: 2025-08-03\\
        \normalsize Status: Active}
        
\date{Version 3.1 — August 2025}

% ============================================
% Document Begin
% ============================================

\begin{document}
\frontmatter
\maketitle

% --------- Executive Summary ---------
\clearpage
\thispagestyle{empty}
\vspace*{2in}
\begin{center}
\Large\bfseries Executive Summary
\end{center}

\noindent
The Pattern Validation Protocol (\PVP{}) v3.1 represents a complete methodology for validating pattern discoveries across all domains and scales. This major release transforms \PVP{} from conceptual framework to practical implementation with:

\begin{itemize}
\item \textbf{Statistical Power Templates}: Domain-specific procedures for effect size and sample calculations
\item \textbf{Computational Complexity Guidance}: Big-O analysis and pragmatic optimization strategies
\item \textbf{Uncertainty Quantification}: Full confidence interval tracking through all stages
\item \textbf{Negative Control Library}: Standardized false patterns for calibration
\item \textbf{Resource-Adaptive Variants}: Multiple protocol versions for different constraints
\item \textbf{Complete Worked Examples}: Known-true, known-false, and ambiguous patterns
\end{itemize}

\vspace{1em}
\noindent
\large\itshape
``Truth emerges not from certainty, but from the rigorous quantification of uncertainty.''

\clearpage
\tableofcontents
\listoffigures
\listoftables
\listofalgorithms

% ============================================
% Main Matter
% ============================================
\mainmatter

\part{Core Protocol}

\chapter{Introduction and Purpose}

\section{Mission Statement}

The Pattern Validation Protocol (\PVP{}) systematizes and validates pattern discovery across all domains and scales using LLM-assisted or hybrid cognition. Version 3.0 transforms \PVP{} from conceptual framework to practical implementation with full statistical rigor, computational guidance, and uncertainty quantification.

\section{Major Upgrades in v3.0}

This major release includes:

\begin{itemize}
\item[\checkmark] \textbf{Statistical Power Implementation Templates}: Domain-specific procedures for effect size and sample calculations
\item[\checkmark] \textbf{Parameter Selection Validation}: Methods for identifying control parameters in novel systems
\item[\checkmark] \textbf{Computational Complexity Guidance}: Big-O analysis and pragmatic shortcuts
\item[\checkmark] \textbf{Uncertainty Propagation Framework}: Confidence interval tracking through all stages
\item[\checkmark] \textbf{Multi-label Pattern Classification}: Hierarchical system for overlapping pattern types
\item[\checkmark] \textbf{Negative Control Library}: Standardized false patterns for calibration
\item[\checkmark] \textbf{Resource-Adaptive Variants}: Multiple protocol versions for different constraints
\end{itemize}

\section{Protocol Philosophy}

\begin{principle}[Rigorous Falsification]
The protocol prioritizes falsification over confirmation. Every pattern must survive multiple attempts at disproof before acceptance.
\end{principle}

\begin{principle}[Computational Feasibility]
All procedures must be computationally tractable. Where exact solutions are intractable, the protocol provides validated approximations with bounded error.
\end{principle}

\begin{principle}[Semantic Stability]
Pattern definitions must remain stable throughout validation. Semantic drift invalidates results.
\end{principle}

\chapter{Full Protocol Overview}

\section{Protocol Stages}

\begin{table}[h]
\centering
\caption{Complete \PVP{} Stage Reference}
\label{tab:stages}
\begin{tabular}{llllc}
\toprule
\textbf{Stage} & \textbf{Name} & \textbf{Canon} & \textbf{Function} & \textbf{Complexity} \\
\midrule
0 & Pattern Hypothesis Registration & I & Define pattern with power analysis & $\BigO(1)$ \\
0.5 & Semantic Anchoring & I & Lock definitions with tolerance & $\BigO(n)$ \\
1 & Observation and Recognition & I & Neutral extraction of motifs & $\BigO(n \log n)$ \\
2 & Critical Deconstruction & I & Red team analysis + artifact check & $\BigO(n)$ \\
2.3 & Qualia Contamination Check & I/II & Optional for consciousness claims & $\BigO(1)$ \\
3 & Null Hypothesis Testing & I & Shuffling, injection, collapse testing & $\BigO(n^2)$ \\
3.5 & Positive Control Injection & I & Synthetic pattern validation & $\BigO(n)$ \\
3.6 & Phase Transition Detection & I/II & Critical threshold mapping & $\BigO(n^p)$ \\
4 & Mechanism Identification & II & Seek lawful or generative explanation & $\BigO(n)$ \\
4.5 & Temporal Validation & I & Time-scale stability analysis & $\BigO(nt)$ \\
5 & Synthesis and Reporting & I/II & Pattern summary, confidence, questions & $\BigO(1)$ \\
6 & Meta-Reflection & III & Bias check, drift analysis, error taxonomy & $\BigO(1)$ \\
$\circlearrowleft$ & Meta-PVP Validation & I/III & Apply protocol to itself & $\BigO(n^2)$ \\
\bottomrule
\end{tabular}
\end{table}

Where $n$ = data size, $p$ = parameter dimensions, $t$ = time points.

\section{Resource-Adaptive Variants}

\subsection{PVP-Micro: Minimal Resource Version}
\begin{itemize}
\item 3 stages: Define \& Hypothesize, Test \& Falsify, Report with Caveats
\item Use when: Extreme resource constraints
\item Validity: Low confidence, requires follow-up
\end{itemize}

\subsection{PVP-Lite: Essential Version}
\begin{itemize}
\item 5 stages: Core validation without advanced tests
\item Use when: Limited resources but need reasonable confidence
\item Validity: Moderate confidence
\end{itemize}

\subsection{PVP-Standard: Recommended Version}
\begin{itemize}
\item 9 stages: Includes all essential stages
\item Use when: Standard research resources available
\item Validity: Moderate to high confidence
\end{itemize}

\subsection{PVP-Complete: Full Protocol}
\begin{itemize}
\item 13+ stages: All stages including optional ones
\item Use when: High-stakes patterns, publication targets
\item Validity: Highest achievable confidence
\end{itemize}

\chapter{Statistical Foundations}

\section{Power Analysis Templates}

\begin{lstlisting}[caption={Domain-Specific Power Analysis},label={lst:power}]
class PowerAnalysisTemplate:
    """Base class for domain-specific power calculations"""
    
    def __init__(self, domain_type, effect_size_priors):
        self.domain = domain_type
        self.priors = effect_size_priors
        self.correction_method = self._select_correction()
    
    def calculate_required_n(self, alpha=0.05, power=0.80, scales=3):
        """Calculate sample size with multiple comparison correction"""
        # Bonferroni correction for multiple scales
        alpha_corrected = alpha / scales
        
        # Domain-specific calculations
        if self.domain == "PHYSICAL":
            return self._physical_n_calculation(alpha_corrected, power)
        elif self.domain == "BIOLOGICAL":
            return self._biological_n_calculation(alpha_corrected, power)
        elif self.domain == "COGNITIVE":
            return self._cognitive_n_calculation(alpha_corrected, power)
        elif self.domain == "CONSCIOUSNESS":
            return self._consciousness_n_calculation(alpha_corrected, power)
    
    def _physical_n_calculation(self, alpha, power):
        """High precision requirements, small effect sizes expected"""
        from statsmodels.stats.power import tt_ind_solve_power
        effect_size = self.priors.get('effect_size', 0.2)  # Small default
        n = tt_ind_solve_power(effect_size=effect_size, 
                               alpha=alpha, 
                               power=power, 
                               alternative='two-sided')
        return int(np.ceil(n * 1.2))  # 20% safety margin
\end{lstlisting}

\section{Effect Size Estimation}

For novel domains where effect sizes are unknown:

\begin{lstlisting}[caption={Conservative Effect Size Estimation}]
def estimate_novel_domain_effect_size(pilot_data, bootstrap_n=1000):
    """Conservative effect size estimation for unknown domains"""
    if len(pilot_data) < 30:
        warnings.warn("Pilot data insufficient; using ultra-conservative priors")
        return 0.1  # Ultra-small effect assumption
    
    # Bootstrap confidence intervals
    effect_sizes = []
    for _ in range(bootstrap_n):
        sample = np.random.choice(pilot_data, size=len(pilot_data), replace=True)
        effect_sizes.append(calculate_cohens_d(sample))
    
    # Use lower bound of 95% CI for conservative estimation
    return np.percentile(effect_sizes, 2.5)
\end{lstlisting}

\section{Multiple Testing Corrections}

\begin{table}[h]
\centering
\caption{Correction Method Selection}
\begin{tabular}{lll}
\toprule
\textbf{Scenario} & \textbf{Method} & \textbf{When to Use} \\
\midrule
Independent tests & Bonferroni & Conservative, few tests \\
Many correlated tests & FDR (Benjamini-Hochberg) & Large-scale testing \\
Unknown correlation & Permutation-based & Computational resources available \\
Hierarchical tests & Hierarchical FDR & Nested hypotheses \\
\bottomrule
\end{tabular}
\end{table}

\chapter{Stage-by-Stage Implementation}

\section{Stage 0: Pattern Hypothesis Registration}

\begin{protocol}[Pattern Registration]
Define the hypothesized pattern with complete specifications:
\begin{enumerate}
\item Mathematical formulation in symbolic form
\item Hierarchical multi-label classification
\item Clear confirmation and refutation criteria
\item Complete power analysis with domain-specific adjustments
\item Prior plausibility rating with uncertainty
\item Computational complexity estimate
\item Resource requirements specification
\end{enumerate}
\end{protocol}

\subsection{Classification System}

\begin{figure}[h]
\centering
\begin{tikzpicture}[
    level 1/.style={sibling distance=4cm},
    level 2/.style={sibling distance=2cm}
]
\node {Pattern}
    child {node {Fundamental Type}
        child {node {Deterministic}}
        child {node {Probabilistic}}
        child {node {Emergent}}
    }
    child {node {Mechanism Type}
        child {node {Algorithmic}}
        child {node {Dynamical}}
        child {node {Informational}}
    }
    child {node {Domain Type}
        child {node {Physical}}
        child {node {Biological}}
        child {node {Cognitive}}
    };
\end{tikzpicture}
\caption{Hierarchical Pattern Classification}
\end{figure}

\section{Stage 0.5: Semantic Anchoring}

\begin{definition}[Triple-Lock Method]
Each key term must be defined three ways:
\begin{enumerate}
\item \textbf{Mathematical}: Formal symbolic representation
\item \textbf{Operational}: Measurement procedure with error bounds
\item \textbf{Invariant}: What remains constant across systems
\end{enumerate}
\end{definition}

\subsection{Domain-Specific Tolerances}

\begin{table}[h]
\centering
\caption{Semantic Dissonance Tolerances by Domain}
\begin{tabular}{lc}
\toprule
\textbf{Domain} & \textbf{Maximum Drift Tolerance} \\
\midrule
Mathematical & 1\% \\
Physical & 5\% \\
Biological & 10\% \\
Cognitive & 15\% \\
Consciousness & 20\% \\
\bottomrule
\end{tabular}
\end{table}

\section{Stage 1: Observation and Recognition}

\begin{algorithm}
\caption{Pattern Recognition with Complexity Management}
\begin{algorithmic}[1]
\Procedure{RecognizePattern}{data, pattern\_type}
    \If{$|data| < 10^4$}
        \State result $\gets$ ExactPatternMatch(data, pattern\_type)
    \ElsIf{$|data| < 10^6$}
        \State chunks $\gets$ ChunkData(data, size=$\sqrt{|data|}$)
        \State result $\gets$ ParallelMatch(chunks, pattern\_type)
    \Else
        \State sample $\gets$ AdaptiveSample(data)
        \State result $\gets$ ApproximateMatch(sample, pattern\_type)
        \State confidence $\gets$ BootstrapConfidence(result, sample)
    \EndIf
    \State \Return result, confidence
\EndProcedure
\end{algorithmic}
\end{algorithm}

\section{Stage 2: Critical Deconstruction}

\begin{protocol}[Red Team Analysis]
Systematically attempt to discredit the pattern:
\begin{enumerate}
\item List all simpler explanations
\item Identify potential measurement artifacts
\item Check for selection bias in data
\item Test sensitivity to analysis choices
\item Search for confounding variables
\item Apply Occam's razor rigorously
\end{enumerate}
\end{protocol}

\section{Stage 3: Null Hypothesis Testing}

\subsection{Null Generation Methods}

\begin{lstlisting}[caption={Comprehensive Null Testing Suite}]
def comprehensive_null_test(data, pattern, max_perms=10000):
    """Test pattern against multiple null models"""
    
    null_methods = {
        'shuffle': lambda d: np.random.permutation(d),
        'phase_random': lambda d: phase_randomize(d),
        'surrogate': lambda d: generate_iaaft_surrogate(d),
        'bootstrap': lambda d: bootstrap_null(d),
        'markov': lambda d: markov_null_model(d)
    }
    
    results = {}
    for method_name, method_func in null_methods.items():
        p_values = []
        
        for _ in range(min(max_perms, 1000)):
            null_data = method_func(data)
            null_pattern = extract_pattern(null_data)
            
            # Compare to observed pattern
            p_values.append(compare_patterns(pattern, null_pattern))
        
        results[method_name] = {
            'p_value': np.mean(p_values),
            'ci_lower': np.percentile(p_values, 2.5),
            'ci_upper': np.percentile(p_values, 97.5)
        }
    
    # Pattern must survive ALL null tests
    return all(r['p_value'] < 0.05 for r in results.values())
\end{lstlisting}

\section{Stage 3.5: Positive Control Injection}

\begin{protocol}[Synthetic Pattern Validation]
\begin{enumerate}
\item Generate synthetic data with known pattern strength
\item Apply full detection pipeline
\item Verify recovery within 10\% of injected strength
\item Test at multiple signal-to-noise ratios
\item Confirm detection threshold matches theoretical predictions
\end{enumerate}
\end{protocol}

\section{Stage 3.6: Phase Transition Detection}

\subsection{Parameter Space Exploration}

\begin{algorithm}
\caption{Adaptive Phase Space Exploration}
\begin{algorithmic}[1]
\Procedure{ExplorePhaseSpace}{parameters, budget}
    \If{$|parameters| \leq 3$}
        \State grid $\gets$ FullFactorialGrid(parameters)
    \Else
        \State \Comment{High-dimensional - use adaptive sampling}
        \State initial $\gets$ LatinHypercube(parameters, $n=10 \cdot |parameters|$)
        \State model $\gets$ GaussianProcess(initial)
        \While{budget $> 0$}
            \State next\_point $\gets$ MaximizeAcquisition(model)
            \State result $\gets$ Evaluate(next\_point)
            \State model.Update(next\_point, result)
            \State budget $\gets$ budget - 1
        \EndWhile
    \EndIf
    \State boundaries $\gets$ IdentifyPhaseBoundaries(model)
    \State \Return boundaries
\EndProcedure
\end{algorithmic}
\end{algorithm}

\section{Stage 4: Mechanism Identification}

\begin{definition}[Valid Mechanism]
A proposed mechanism must:
\begin{enumerate}
\item Generate the observed pattern when simulated
\item Be eliminated when the mechanism is blocked
\item Make testable predictions beyond the original pattern
\item Be consistent with known physical/biological laws
\item Be more parsimonious than alternative explanations
\end{enumerate}
\end{definition}

\section{Stage 4.5: Temporal Validation}

\begin{lstlisting}[caption={Temporal Stability Analysis}]
def temporal_validation(time_series, pattern_extractor):
    """Validate pattern stability across timescales"""
    
    # Bootstrap timescales from data
    timescales = estimate_relevant_timescales(time_series)
    
    stability_results = {}
    for scale in np.logspace(
        np.log10(timescales['min']),
        np.log10(timescales['max']),
        num=20
    ):
        windows = sliding_window(time_series, window_size=scale)
        patterns = [pattern_extractor(w) for w in windows]
        
        stability_results[scale] = {
            'mean': np.mean(patterns, axis=0),
            'std': np.std(patterns, axis=0),
            'cv': np.std(patterns, axis=0) / np.mean(patterns, axis=0),
            'stationary': test_stationarity(patterns)
        }
    
    # Find minimum stable timescale
    stable_scales = [s for s, r in stability_results.items() 
                     if r['cv'] < 0.1 and r['stationary']]
    
    return {
        'min_stable_scale': min(stable_scales) if stable_scales else None,
        'all_results': stability_results
    }
\end{lstlisting}

\section{Stage 5: Synthesis and Reporting}

\begin{protocol}[Structured Reporting Requirements]
Every \PVP{} report must include:
\begin{enumerate}
\item \textbf{Executive Summary}: Pass/fail, confidence, key insight, action items
\item \textbf{Pattern Details}: Final formulation, classification, validity domains
\item \textbf{Statistical Summary}: Effect sizes, power achieved, sample sizes
\item \textbf{Validation Results}: Stage-by-stage outcomes with scores
\item \textbf{Proposed Experiments}: Falsification, mechanism, and boundary tests
\item \textbf{Open Questions}: Unresolved issues and required resources
\item \textbf{Code \& Data}: Repository links and computational notebooks
\end{enumerate}
\end{protocol}

\section{Stage 6: Meta-Reflection}

\subsection{Error Mode Taxonomy}

\begin{table}[h]
\centering
\caption{Comprehensive Error Mode Classification}
\begin{tabular}{L{2.5cm}L{8cm}}
\toprule
\textbf{Category} & \textbf{Common Failure Modes} \\
\midrule
Statistical & Overfitting, p-hacking, selection bias, multiple testing errors \\
Semantic & Semantic drift, category errors, metaphor literalization \\
Methodological & Control collapse, temporal aliasing, scale conflation \\
Computational & Intractability, approximation cascade, precision loss \\
Cognitive & Anthropic projection, pareidolia, confirmation seeking \\
Emergent & Artifactual emergence, observer effect, complexity collapse \\
\bottomrule
\end{tabular}
\end{table}

\chapter{Uncertainty Quantification}

\section{Uncertainty Propagation Framework}

\begin{lstlisting}[caption={Complete Uncertainty Propagation}]
class UncertaintyPropagator:
    """Track confidence intervals through all stages"""
    
    def __init__(self):
        self.stage_uncertainties = {}
        
    def propagate(self, stage_name, input_uncertainty, stage_function):
        """Propagate uncertainty through a stage using Monte Carlo"""
        
        output_samples = []
        for _ in range(1000):
            # Sample from input distribution
            input_sample = sample_from_uncertainty(input_uncertainty)
            # Apply stage function
            output = stage_function(input_sample)
            output_samples.append(output)
        
        # Calculate output uncertainty
        output_uncertainty = {
            'mean': np.mean(output_samples),
            'std': np.std(output_samples),
            'ci_lower': np.percentile(output_samples, 2.5),
            'ci_upper': np.percentile(output_samples, 97.5)
        }
        
        self.stage_uncertainties[stage_name] = output_uncertainty
        return output_uncertainty
    
    def get_final_confidence_interval(self):
        """Aggregate uncertainties across all stages"""
        # Account for correlations between stages
        correlation_matrix = estimate_stage_correlations(
            self.stage_uncertainties
        )
        
        # Use multivariate normal for correlated uncertainties
        aggregated = multivariate_aggregate(
            self.stage_uncertainties,
            correlation_matrix
        )
        
        return aggregated
\end{lstlisting}

\section{Confidence Drift Monitoring}

\begin{figure}[h]
\centering
\begin{tikzpicture}
\begin{axis}[
    xlabel={Stage},
    ylabel={Confidence (\%)},
    xmin=0, xmax=7,
    ymin=0, ymax=100,
    xtick={1,2,3,4,5,6},
    xticklabels={S1,S2,S3,S4,S5,S6},
    legend pos=south east,
    grid=major
]

% Normal pattern
\addplot[color=passcolor, mark=o, thick] coordinates {
    (1,75) (2,72) (3,78) (4,82) (5,85) (6,83)
};
\addlegendentry{Normal Drift}

% Concerning pattern
\addplot[color=warncolor, mark=square, thick] coordinates {
    (1,80) (2,65) (3,45) (4,72) (5,38) (6,25)
};
\addlegendentry{Concerning Drift}

% Threshold
\addplot[color=red, dashed, thick] coordinates {
    (0,50) (7,50)
};
\addlegendentry{Threshold}

\end{axis}
\end{tikzpicture}
\caption{Confidence Drift Patterns}
\end{figure}

\part{Computational Optimization}

\chapter{Computational Complexity Analysis}

\section{Stage-by-Stage Complexity}

\begin{table}[h]
\centering
\caption{Detailed Computational Complexity by Stage}
\begin{tabular}{llllll}
\toprule
\textbf{Stage} & \textbf{Best} & \textbf{Average} & \textbf{Worst} & \textbf{Memory} & \textbf{Optimization} \\
\midrule
0 & $\BigO(1)$ & $\BigO(1)$ & $\BigO(1)$ & $\BigO(1)$ & Pre-compute templates \\
1 & $\BigO(n)$ & $\BigO(n \log n)$ & $\BigO(n^2)$ & $\BigO(n)$ & Sliding windows \\
2 & $\BigO(1)$ & $\BigO(n)$ & $\BigO(n)$ & $\BigO(1)$ & Parallelize validators \\
3 & $\BigO(n)$ & $\BigO(n^2)$ & $\BigO(n^3)$ & $\BigO(n)$ & Monte Carlo sampling \\
3.6 & $\BigO(p^2)$ & $\BigO(n^p)$ & $\BigO(n^p)$ & $\BigO(n^p)$ & Dimension reduction \\
4.5 & $\BigO(t)$ & $\BigO(nt)$ & $\BigO(nt^2)$ & $\BigO(nt)$ & Downsample if $t > 10^6$ \\
\bottomrule
\end{tabular}
\end{table}

\section{Memory Requirements}

\begin{table}[h]
\centering
\caption{Memory Requirements by Data Scale}
\begin{tabular}{lrrrrr}
\toprule
\textbf{Data Points} & \textbf{Stage 1} & \textbf{Stage 3} & \textbf{Stage 3.6} & \textbf{Stage 4.5} & \textbf{Peak} \\
\midrule
$10^3$ & 8 KB & 8 MB & 1 MB & 100 KB & $\sim$10 MB \\
$10^4$ & 80 KB & 800 MB & 10 MB & 1 MB & $\sim$1 GB \\
$10^5$ & 800 KB & 80 GB* & 100 MB & 10 MB & $\sim$80 GB \\
$10^6$ & 8 MB & 8 TB* & 1 GB & 100 MB & $\sim$1 GB** \\
$10^7$ & 80 MB & 800 TB* & 10 GB & 1 GB & $\sim$10 GB** \\
\bottomrule
\end{tabular}
\end{table}
*Without optimization **With streaming/chunking

\section{Optimization Strategies}

\subsection{Parallelization Opportunities}

\begin{lstlisting}[caption={Parallel Execution Framework}]
from concurrent.futures import ProcessPoolExecutor, ThreadPoolExecutor
import multiprocessing as mp

class ParallelPVP:
    """Parallel execution strategies for PVP"""
    
    @staticmethod
    def parallel_validation(data_chunks, validation_func, n_workers=None):
        """Parallel validation across data chunks"""
        
        if n_workers is None:
            n_workers = mp.cpu_count() - 1
        
        with ProcessPoolExecutor(max_workers=n_workers) as executor:
            futures = [
                executor.submit(validation_func, chunk)
                for chunk in data_chunks
            ]
            
            results = []
            for future in futures:
                try:
                    results.append(future.result(timeout=300))
                except TimeoutError:
                    results.append({'error': 'timeout'})
        
        return merge_validation_results(results)
\end{lstlisting}

\subsection{GPU Acceleration}

\begin{lstlisting}[caption={GPU-Accelerated Permutation Testing}]
def gpu_accelerated_permutation(data, statistic_func, n_perms=10000):
    """GPU acceleration for massive permutation tests"""
    try:
        import cupy as cp
        
        # Transfer to GPU
        data_gpu = cp.array(data)
        
        # Generate all permutations on GPU
        perms = cp.random.permutation(
            cp.tile(data_gpu, (n_perms, 1))
        )
        
        # Vectorized statistic calculation
        perm_stats = statistic_func(perms, axis=1)
        observed = statistic_func(data_gpu)
        
        # P-value calculation
        p_value = (perm_stats >= observed).mean()
        
        return float(p_value)
        
    except ImportError:
        print("GPU acceleration unavailable, falling back to CPU")
        return None
\end{lstlisting}

\section{Optimization Decision Tree}

\begin{figure}[h]
\centering
\begin{tikzpicture}[
    decision/.style={diamond, draw, aspect=2, inner sep=0pt, minimum width=3cm},
    process/.style={rectangle, draw, rounded corners},
    optimize/.style={rectangle, draw, fill=blue!20},
    scale=0.9
]

\node[decision] (size) {Data > $10^6$?};
\node[process, below left=2cm of size] (small) {Standard Processing};
\node[decision, below right=2cm of size] (memory) {Memory > Required?};
\node[optimize, below left=2cm of memory] (stream) {Use Streaming};
\node[decision, below right=2cm of memory] (gpu) {GPU Available?};
\node[optimize, below left=2cm of gpu] (sample) {Use Sampling};
\node[optimize, below right=2cm of gpu] (accelerate) {GPU Accelerate};

\draw[->] (size) -- node[left] {No} (small);
\draw[->] (size) -- node[right] {Yes} (memory);
\draw[->] (memory) -- node[left] {No} (stream);
\draw[->] (memory) -- node[right] {Yes} (gpu);
\draw[->] (gpu) -- node[left] {No} (sample);
\draw[->] (gpu) -- node[right] {Yes} (accelerate);

\end{tikzpicture}
\caption{Optimization Strategy Selection}
\end{figure}

\part{Validation Libraries}

\chapter{Negative Control Library}

\section{Purpose and Usage}

The Negative Control Library provides validated false patterns for protocol calibration. These patterns are known to be false but may appear real under naive analysis.

\section{Standard False Patterns}

\begin{table}[h]
\centering
\caption{Negative Control Patterns}
\begin{tabular}{L{3cm}L{2cm}L{4cm}c}
\toprule
\textbf{Pattern} & \textbf{Domain} & \textbf{Why False} & \textbf{Fails at Stage} \\
\midrule
Numerological Birthday & Cognitive & No mechanism, fails randomization & 3 \\
Mars Retrograde Markets & Physical/Economic & No causal mechanism & 4 \\
Pyramid Power & Physical & No reproducible effect & 3.5 \\
Bible Code & Information & Same in any large text & 3 \\
Random Walk Consciousness & Physical/Cognitive & Statistical artifacts & 2 \\
\bottomrule
\end{tabular}
\end{table}

\section{Synthetic False Pattern Generation}

\begin{lstlisting}[caption={Generate Calibration Patterns}]
class NegativeControlGenerator:
    """Generate synthetic false patterns for testing"""
    
    @staticmethod
    def generate_spurious_correlation(n=1000, correlation=0.3):
        """Create data with spurious correlation"""
        # Generate independent variables
        x = np.random.normal(0, 1, n)
        y = np.random.normal(0, 1, n)
        
        # Add confounding variable
        confounder = np.random.normal(0, 1, n)
        
        # Create spurious correlation through confounder
        x_observed = x + correlation * confounder
        y_observed = y + correlation * confounder
        
        # Will show correlation but no causal relationship
        return x_observed, y_observed, confounder
    
    @staticmethod
    def generate_selection_bias_pattern(n=10000, bias_strength=0.5):
        """Create pattern that only exists due to selection bias"""
        # Generate random data
        data = np.random.normal(0, 1, n)
        
        # Select biased subset
        threshold = np.percentile(data, 100 * (1 - bias_strength))
        selected = data[data > threshold]
        
        # Pattern appears in selected data but not in full dataset
        return data, selected
\end{lstlisting}

\chapter{Pattern Classification System}

\section{Hierarchical Multi-Label Taxonomy}

\begin{figure}[h]
\centering
\begin{tikzpicture}[
    every node/.style={rectangle, draw, rounded corners, minimum width=2.5cm},
    fundamental/.style={fill=blue!20},
    mechanism/.style={fill=green!20},
    domain/.style={fill=orange!20},
    level distance=2cm,
    sibling distance=3cm
]

\node[fundamental] {Pattern}
    child {node[fundamental] {Fundamental}
        child {node {Deterministic}
            child {node {Mechanical}}
            child {node {Mathematical}}
            child {node {Logical}}
        }
        child {node {Probabilistic}
            child {node {Statistical}}
            child {node {Stochastic}}
            child {node {Quantum}}
        }
        child {node {Emergent}
            child {node {Weak}}
            child {node {Strong}}
            child {node {Computational}}
        }
    }
    child {node[mechanism] {Mechanism}
        child {node {Algorithmic}}
        child {node {Dynamical}}
        child {node {Informational}}
    }
    child {node[domain] {Domain}
        child {node {Physical}}
        child {node {Biological}}
        child {node {Cognitive}}
        child {node {Narrative}}
    };

\end{tikzpicture}
\caption{Complete Pattern Classification Hierarchy}
\end{figure}

\section{Classification Rules}

\begin{protocol}[Multi-Label Classification]
\begin{enumerate}
\item Patterns can have multiple labels within each category
\item All three categories (Fundamental, Mechanism, Domain) must be specified
\item Labels should be ordered by relevance/strength
\item Justification required for each label assignment
\item Cross-category constraints must be respected (e.g., quantum requires physical domain)
\end{enumerate}
\end{protocol}

\part{Worked Examples}

\chapter{Example 1: Known-True Pattern}
\section{Neural Avalanches and Criticality}

\subsection{Pattern Overview}
Neural avalanches in cortical networks follow a power-law distribution with critical exponent $\alpha \approx 1.5$, indicating the brain operates near a critical point for optimal information processing.

\subsection{Stage 0: Registration}

\begin{lstlisting}[caption={Pattern Definition}]
# Mathematical formulation
P(s) = s^(-alpha) where alpha = 1.5
# P(s) is probability of avalanche size s

# Classification
fundamental_type = ['Probabilistic', 'Emergent']
mechanism_type = ['Dynamical', 'Informational']
domain = ['Biological', 'Cognitive']

# Power analysis
required_n = 156  # avalanches per condition
# Based on effect size = 0.5, CV = 0.4, power = 0.80
\end{lstlisting}

\subsection{Stage 1-2: Observation and Deconstruction}

\begin{lstlisting}[caption={Initial Analysis}]
# Load and process data
avalanches = detect_avalanche(spike_times, threshold_ms=5)
sizes = [len(av) for av in avalanches]

# Results:
# Total avalanches: 15,234
# Size range: 1 - 847
# Heavy-tailed distribution observed

# Critical deconstruction checks:
# 1. Binning artifact test: PASSED (CV < 10%)
# 2. Electrode spacing: Adequate coverage confirmed
# 3. Non-stationarity: Controlled via windowing
\end{lstlisting}

\subsection{Stage 3: Null Hypothesis Testing}

\begin{table}[h]
\centering
\caption{Null Test Results}
\begin{tabular}{lcc}
\toprule
\textbf{Null Method} & \textbf{Preserves Power Law} & \textbf{Result} \\
\midrule
Shuffle & No & \textcolor{passcolor}{PASS} \\
Poisson & No & \textcolor{passcolor}{PASS} \\
Surrogate & No & \textcolor{passcolor}{PASS} \\
\bottomrule
\end{tabular}
\end{table}

\subsection{Stage 3.6: Phase Transition}

\begin{figure}[h]
\centering
\begin{tikzpicture}
\begin{axis}[
    xlabel={Excitation (E)},
    ylabel={Inhibition (I)},
    zlabel={$\alpha$},
    width=10cm,
    height=8cm,
    view={45}{30},
    colormap/cool,
    colorbar
]
\addplot3[
    surf,
    samples=20,
    domain=0.5:2
] {1.5 + 0.3*sin(deg(x*pi)) * cos(deg(y*pi))};
\end{axis}
\end{tikzpicture}
\caption{Phase Diagram: Critical exponent $\alpha$ as function of E/I balance}
\end{figure}

\subsection{Final Assessment}

\begin{itemize}
\item \textbf{Pattern Status}: \textcolor{passcolor}{CONFIRMED}
\item \textbf{Confidence}: 92\% (CI: 88\%-95\%)
\item \textbf{Key Finding}: Brain operates near $E/I \approx 1$ for criticality
\item \textbf{Mechanism}: Self-organized criticality through synaptic plasticity
\end{itemize}

\chapter{Example 2: Known-False Pattern}
\section{Astrological Market Prediction}

\subsection{Pattern Claim}
Mercury retrograde periods predict increased probability of stock market crashes.

\subsection{Initial Observation}
\begin{lstlisting}[caption={Spurious Correlation}]
# Initial analysis shows apparent effect:
P(crash|retrograde) = 0.023
P(crash|normal) = 0.019
Ratio = 1.21 (21% increase!)

# But this is spurious...
\end{lstlisting}

\subsection{Stage 3: Proper Statistical Testing}

\begin{lstlisting}[caption={Multiple Testing Correction}]
# Uncorrected p-value: 0.042 (seems significant!)

# But we're testing 8 planets x 2 directions = 16 hypotheses
p_corrected = 0.042 * 16 = 0.672  # Not significant

# Permutation test confirms:
p_value_permutation = 0.238  # Pattern within noise
\end{lstlisting}

\subsection{Stage 4: Mechanism Search}

\begin{table}[h]
\centering
\caption{Physical Mechanism Analysis}
\begin{tabular}{lcc}
\toprule
\textbf{Proposed Mechanism} & \textbf{Relative Strength} & \textbf{Plausible?} \\
\midrule
Gravitational & $2.6 \times 10^{-7}$ vs Moon & \textcolor{failcolor}{NO} \\
Electromagnetic & $1.1 \times 10^{-18}$ vs Earth & \textcolor{failcolor}{NO} \\
\bottomrule
\end{tabular}
\end{table}

\subsection{Pattern REJECTED at Stage 4}
\begin{itemize}
\item Initial correlation was spurious
\item Failed proper statistical testing
\item No plausible physical mechanism
\item Positive control (Monday effect) confirms tests work
\end{itemize}

\chapter{Example 3: Ambiguous Pattern}
\section{Meditation-Induced EEG Coherence}

\subsection{Why Ambiguous?}
\begin{enumerate}
\item Some studies show effect, others don't
\item Multiple meditation types exist
\item "Long-term" poorly defined
\item Measurement methods vary
\end{enumerate}

\subsection{Critical: Semantic Anchoring}

\begin{lstlisting}[caption={Ultra-Precise Definitions}]
definitions = {
    'long_term_meditator': {
        'mathematical': 'practice_hours > 10000',
        'operational': 'Daily >2hr for >10 years + retreats',
        'invariant': 'Sustained attention neural changes'
    },
    'gamma_coherence': {
        'mathematical': 'PLV = |E[exp(i*(phi1-phi2))]|',
        'operational': 'Morlet wavelet -> Hilbert -> PLV',
        'invariant': 'Phase relationship, not amplitude'
    }
}
\end{lstlisting}

\subsection{Sensitivity Analysis}

\begin{figure}[h]
\centering
\begin{tikzpicture}
\begin{axis}[
    xlabel={Practice Hours Threshold},
    ylabel={Effect Size (d)},
    xmin=5000, xmax=15000,
    ymin=0, ymax=1,
    legend pos=north east,
    grid=major
]

\addplot[color=blue, mark=o, thick] coordinates {
    (5000,0.35) (7500,0.48) (10000,0.61) (12500,0.72) (15000,0.78)
};
\addlegendentry{Effect Size}

\addplot[color=red, dashed, thick] coordinates {
    (5000,0.5) (15000,0.5)
};
\addlegendentry{d=0.5 threshold}

\end{axis}
\end{tikzpicture}
\caption{Effect Size Sensitivity to Definition}
\end{figure}

\subsection{Resolution: PROVISIONAL}

\begin{itemize}
\item Pattern shows promise but requires:
  \begin{enumerate}
  \item Standardized experience quantification
  \item Multi-site replication protocol
  \item Mechanism test battery
  \item Public raw data repository
  \end{enumerate}
\item Focus on 8,000-12,000 hour practitioners
\item Clear research program defined
\end{itemize}

\part{Implementation Guidance}

\chapter{Quick Start Guide}

\section{Choosing Your PVP Variant}

\begin{figure}[h]
\centering
\begin{tikzpicture}[
    node distance=2cm,
    decision/.style={diamond, draw, aspect=2},
    variant/.style={rectangle, draw, rounded corners, fill=blue!20}
]

\node[decision] (resources) {Resources?};
\node[variant, below left=of resources] (micro) {PVP-Micro};
\node[decision, below right=of resources] (stakes) {Stakes?};
\node[variant, below left=of stakes] (standard) {PVP-Standard};
\node[variant, below right=of stakes] (complete) {PVP-Complete};

\draw[->] (resources) -- node[left] {Minimal} (micro);
\draw[->] (resources) -- node[right] {Available} (stakes);
\draw[->] (stakes) -- node[left] {Normal} (standard);
\draw[->] (stakes) -- node[right] {High} (complete);

\end{tikzpicture}
\caption{PVP Variant Selection}
\end{figure}

\section{Pre-Validation Checklist}

\begin{itemize}
\item[$\square$] Pattern mathematically defined
\item[$\square$] Power analysis completed
\item[$\square$] Semantic anchors locked
\item[$\square$] Computational budget estimated
\item[$\square$] Negative controls selected
\item[$\square$] Data quality verified
\item[$\square$] Analysis code tested on synthetic data
\end{itemize}

\section{Red Flags: When to Stop}

\begin{warning}[Critical Failure Points]
Stop validation immediately if:
\begin{itemize}
\item Semantic drift exceeds domain threshold
\item Effect only visible with one specific analysis
\item Pattern requires increasingly complex explanations
\item Negative controls show similar patterns
\item Computational requirements become intractable
\end{itemize}
\end{warning}

\chapter{Common Pitfalls and Solutions}

\section{Statistical Pitfalls}

\begin{table}[h]
\centering
\caption{Common Statistical Errors and Remedies}
\begin{tabular}{L{3cm}L{4cm}L{4cm}}
\toprule
\textbf{Pitfall} & \textbf{Consequence} & \textbf{Solution} \\
\midrule
P-hacking & False positives & Pre-register all analyses \\
Overfitting & Poor generalization & Use held-out validation set \\
Multiple testing & Inflated Type I error & Apply appropriate corrections \\
Small sample & Unreliable results & Use power analysis upfront \\
\bottomrule
\end{tabular}
\end{table}

\section{Computational Pitfalls}

\begin{lstlisting}[caption={Memory Management Example}]
# BAD: Loading everything into memory
def process_large_dataset(filename):
    data = np.load(filename)  # May cause memory error
    return analyze(data)

# GOOD: Streaming processing
def process_large_dataset_streaming(filename):
    results = []
    with h5py.File(filename, 'r') as f:
        for chunk in chunks(f['data'], size=10000):
            results.append(analyze(chunk))
    return aggregate(results)
\end{lstlisting}

\section{Semantic Pitfalls}

\begin{protocol}[Preventing Semantic Drift]
\begin{enumerate}
\item Document all definitions at Stage 0.5
\item Create semantic fingerprints of key terms
\item Monitor drift at each stage
\item Flag any change > 5\%
\item Re-anchor if drift detected
\item Document all definition changes
\end{enumerate}
\end{protocol}

\chapter{Meta-Validation Protocol}

\section{Self-Application of PVP}

The PVP must validate itself through:

\begin{enumerate}
\item \textbf{Negative Control Performance}: Correctly reject 95\%+ of known-false patterns
\item \textbf{Positive Control Sensitivity}: Detect 90\%+ of synthetic patterns
\item \textbf{Cross-Scale Validation}: Work across 3+ scales
\item \textbf{Temporal Stability}: Consistent results over 6 months
\item \textbf{Computational Efficiency}: Meet stated complexity bounds
\item \textbf{Inter-Team Agreement}: 3 independent teams converge
\end{enumerate}

\section{Certification Flow}

\begin{figure}[h]
\centering
\begin{tikzpicture}[
    node distance=1.5cm,
    process/.style={rectangle, draw, rounded corners},
    pass/.style={process, fill=green!20},
    scale=0.9
]

\node[process] (neg) {Negative Controls};
\node[process, below=of neg] (pos) {Positive Controls};
\node[process, below=of pos] (border) {Borderline Cases};
\node[process, below=of border] (hist) {Historical Test};
\node[process, below=of hist] (live) {Live Discovery};
\node[process, below=of live] (comp) {Computational};
\node[process, below=of comp] (temp) {Temporal};
\node[pass, below=of temp] (cert) {Certified v3.0};

\draw[->] (neg) -- node[right] {15/15 rejected} (pos);
\draw[->] (pos) -- node[right] {10/10 detected} (border);
\draw[->] (border) -- node[right] {7/10 + justified} (hist);
\draw[->] (hist) -- node[right] {5/5 rejected} (live);
\draw[->] (live) -- node[right] {replicated} (comp);
\draw[->] (comp) -- node[right] {meets bounds} (temp);
\draw[->] (temp) -- node[right] {6-month stable} (cert);

\end{tikzpicture}
\caption{Meta-PVP Certification Flow}
\end{figure}

% ============================================
% Back Matter
% ============================================
\backmatter

\appendix

\chapter{Statistical Tables and Formulas}

\section{Effect Size Calculations}

\begin{align}
\text{Cohen's } d &= \frac{\bar{X}_1 - \bar{X}_2}{s_{\text{pooled}}} \\
s_{\text{pooled}} &= \sqrt{\frac{(n_1-1)s_1^2 + (n_2-1)s_2^2}{n_1+n_2-2}}
\end{align}

\section{Power Calculation Reference}

\begin{table}[h]
\centering
\caption{Sample Size for 80\% Power (two-tailed, $\alpha=0.05$)}
\begin{tabular}{lccccc}
\toprule
\textbf{Effect Size (d)} & 0.2 & 0.5 & 0.8 & 1.0 & 1.2 \\
\midrule
Sample per group & 394 & 64 & 26 & 17 & 12 \\
\bottomrule
\end{tabular}
\end{table}

\chapter{Code Templates}

\section{Complete PVP Implementation Template}

\begin{lstlisting}[caption={Full PVP Pipeline}]
class PatternValidationPipeline:
    """Complete PVP implementation"""
    
    def __init__(self, pattern_definition, data, variant='standard'):
        self.pattern = pattern_definition
        self.data = data
        self.variant = variant
        self.results = {}
        self.uncertainties = UncertaintyPropagator()
        
    def run_validation(self):
        """Execute full validation pipeline"""
        
        # Stage 0: Registration
        self.register_pattern()
        
        # Stage 0.5: Semantic Anchoring
        if not self.anchor_semantics():
            return self.fail("Semantic instability")
        
        # Stage 1: Observation
        observations = self.observe_pattern()
        
        # Stage 2: Deconstruction
        if not self.critical_analysis(observations):
            return self.fail("Simpler explanation exists")
        
        # Stage 3: Null Testing
        if not self.null_hypothesis_suite():
            return self.fail("Pattern in noise")
        
        # Stage 3.5: Positive Control
        if not self.verify_detection_capability():
            return self.fail("Cannot detect known patterns")
        
        # Optional stages based on variant
        if self.variant in ['standard', 'complete']:
            self.phase_transition_analysis()
            self.mechanism_search()
            
        if self.variant == 'complete':
            self.temporal_validation()
            
        # Stage 5: Synthesis
        self.synthesize_results()
        
        # Stage 6: Meta-reflection
        self.meta_analysis()
        
        return self.results
\end{lstlisting}

\chapter{Supplementary Materials}

\section{Required Software}

\begin{table}[h]
\centering
\caption{Software Dependencies}
\begin{tabular}{lll}
\toprule
\textbf{Package} & \textbf{Version} & \textbf{Purpose} \\
\midrule
NumPy & $\geq$ 1.20 & Numerical computing \\
SciPy & $\geq$ 1.7 & Statistical tests \\
Statsmodels & $\geq$ 0.12 & Power analysis \\
Scikit-learn & $\geq$ 0.24 & Machine learning \\
Matplotlib & $\geq$ 3.4 & Visualization \\
CuPy & Optional & GPU acceleration \\
\bottomrule
\end{tabular}
\end{table}

\section{Data Format Standards}

\begin{lstlisting}[caption={Standard Data Format}]
# HDF5 structure for PVP data
pvp_data.h5
├── metadata/
│   ├── pattern_definition
│   ├── collection_parameters
│   └── preprocessing_steps
├── raw_data/
│   ├── observations
│   └── timestamps
├── processed_data/
│   ├── features
│   └── patterns
└── validation_results/
    ├── stage_outcomes
    └── confidence_intervals
\end{lstlisting}

\section{Contact and Support}

\begin{itemize}
\item \textbf{Methodology Support}: methodology@fractality.institute
\item \textbf{Code Repository}: \url{https://github.com/fractality/pvp-v3}
\item \textbf{Issue Tracking}: \url{https://github.com/fractality/pvp-v3/issues}
\item \textbf{Community Forum}: \url{https://forum.fractality.institute/pvp}
\end{itemize}

\chapter{Version History}

\begin{itemize}
\item \textbf{v3.1} (2025-08-24): RE-RELEASE by Claude Opus 4.1 after completion of FI-UCT-v9.1
\item \textbf{v3.0} (2025-08-03): MAJOR RELEASE - Full implementation with statistics, computation, uncertainty
\item \textbf{v2.3} (2025-08-02): Confidence drift monitoring, pattern classification, error taxonomy  
\item \textbf{v2.2} (2025-08-02): Temporal dynamics, phase transitions, expanded scales
\item \textbf{v2.1} (2025-07-15): Qualia check, meta-validation, auto-falsifier
\item \textbf{v2.0} (2025-06-01): Major architectural revision
\item \textbf{v1.0} (2025-01-15): Initial release
\end{itemize}

% ============================================
% Closing
% ============================================

\clearpage
\thispagestyle{empty}
\vspace*{\fill}
\begin{center}
\Large
\textbf{Pattern Validation Protocol v3.1}\\[1em]
\large
Truth Through Rigorous Testing\\[2em]
\normalsize
\textit{``Truth emerges not from certainty,\\
but from the rigorous quantification of uncertainty.''}\\[3em]
The Fractality Institute\\
\url{https://fractality.institute}
\end{center}
\vspace*{\fill}

\end{document}